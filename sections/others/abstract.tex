\begin{abstractBox}[colbacktitle=black]{Abstract (EN)}{
    For many real-world applications, a system of interest can be represented via a mathematical model which depends on a set of parameters. 
    In order to identify the parameters, a set of observations is available and an Inverse Problem is formulated. 
    Often, identifying the parameters from the observations is often a challenging task, especially when the model is expensive to evaluate. 
    This is the case of Partial Differential Equations models, where numerical simulations which are both inexact and computationally expensive are required to obtain the model output. 
    To ease the computational costs, surrogate models can be used to approximate the model output. 
    In this work, we present two different regression techniques, Gaussian Process Regression and Lipschitz Regression. 
    After reformulating the Inverse Problem to account for the surrogate model, we develope an adaptive training strategy to train the surrogate model. 
    The proposed training strategy aims at optimizing not only the training points' positions but also their evaluation accuracies.
    Moreover, interleaved sampling of the posterior distribution of the unknown parameters is performed while the surrogate model is trained, providing a solution to the Inverse Problem.
    The quality of the surrogating techniques as well as the effectiveness of the adaptive training strategy are tested through different numerical experiments.
}
\end{abstractBox}
\begin{abstractBox}[colbacktitle=black]{Abstract (DE)}{
    Für viele reale Anwendungen kann ein System von Interesse durch ein mathematisches Modell dargestellt werden, das von einer Reihe von Parametern abhängt. 
    Um die Parameter zu ermitteln, ist eine Reihe von Beobachtungen verfügbar, und es wird ein inverses Problem formuliert. 
    Die Identifizierung der Parameter aus den Beobachtungen ist oft eine schwierige Aufgabe, vor allem, wenn das Modell teuer zu bewerten ist. 
    Dies ist bei Modellen für partielle Differentialgleichungen der Fall, bei denen numerische Simulationen, die sowohl ungenau als auch rechenintensiv sind, erforderlich sind, um das Modellergebnis zu erhalten. 
    Um die Rechenkosten zu senken, können Ersatzmodelle verwendet werden, um das Modellergebnis zu approximieren. 
    In dieser Arbeit stellen wir zwei verschiedene Regressionstechniken vor, die Gaußsche Prozessregression und die Lipschitz-Regression. 
    Nach der Neuformulierung des inversen Problems zur Berücksichtigung des Surrogatmodells entwickeln wir eine adaptive Trainingsstrategie, um das Surrogatmodell zu trainieren. 
    Die vorgeschlagene Trainingsstrategie zielt darauf ab, nicht nur die Positionen der Trainingspunkte, sondern auch ihre Auswertungsgenauigkeit zu optimieren.
    Darüber hinaus wird während des Trainings des Surrogatmodells eine verschachtelte Abtastung der posterioren Verteilung der unbekannten Parameter durchgeführt, was eine Lösung des inversen Problems ermöglicht.
    Die Qualität der Surrogattechniken sowie die Wirksamkeit der adaptiven Trainingsstrategie werden durch verschiedene numerische Experimente getestet.
    }
    \end{abstractBox}
\vspace{.5cm}