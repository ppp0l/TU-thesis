\section{Inverse Problems and Uncertainty Quantification} \label{sec:IPUQ}
Inverse Problems (IP) deal with the identification of some unknown parameter or function in a model through observations of the portion of reality the model is intended to represent. \\
A first approach to IP is that of Classical Inverse Problems, where the problem is treated in a deterministic framework, questions of well-posedness are tackled, and the solution is a point estimate of the unknown.
Another possibility is given by the Bayesian approach to IP: the problem is treated as a statistical problem where unknowns are assumed to have random behaviour, and probabilistic tools are utilized intensively. 
In the Bayesian treatment, the problem is usually well-posed, but its solution is a probability distribution requiring a more careful and complex numerical handling.\\
Uncertainty Quantification (UQ) studies how uncertainty propagates from the different variables and components of a mathematical model, and adopts a stochastic perspective on various tasks including IP. 
UQ is not
\subsection{Classical Inverse Problems}\label{sec:IP}
\lipsum