\section{Preliminaries} \label{sec:preliminaries}

In the remainder of this section Section~\ref{sec:BIP} first introduces the basic concepts of Inverse Problems and then presents the Bayesian approach to IP, providing a general framework for the formulation of Bayesian Inverse Problems (BIPs) and the derivation of the posterior distribution of the unknown given the observations; Section~\ref{sec:IP-sol} presents some key techniques for the solution of IPs and their numerical treatment, with a focus on the Bayesian approach; Section~\ref{sec:PDE} introduces the basic concepts of Partial Differential Equation (PDE) models, with a focus on the Laplace equation, the diffusion equation and elastomechanic equations as they are the PDEs used in the numerical experiments of Section~\ref{sec:exp}; finally, Section~\ref{sec:AdaFE} presents the Finite Element (FE) method and its adaptive version, which is the solution method for PDEs considered throughout this work.

\subsection{Bayesian Inverse Problems}\label{sec:BIP}

The main reference for this section and the following is Tim Sullivan's Introduction to Uncertainty Quantification~\cite{Sullivan2015}, and other sources will be quoted when utilized. \medskip

Inverse Problems (IP) deal with the identification of some unknown parameter or function in a model through observations of the portion of reality that the model is intended to represent.\newline
Given a measurement $y^m$ in the measurement space $\mc Y$, also known as observation space, the goal of an IP is to identify the pre-image $p*$ in the parameter space $\Theta$ under a map \[y : \Theta \longrightarrow \mc Y \] known as the forward model or parameter-to-observation map.
If a unique $p*$ in $\Theta$ such that
\begin{equation}\label{eq:IP0}
    y(p*) = y^m
\end{equation}
exists, the IP admits a unique solution: but this case is an exception rather than a rule, especially when the observation $y^m$ is corrupted by noise.

A number of techniques have been developed to address from an analytical perspective the numerous issues which arise in an IP.
First, the problem can be reformulated as a least squares problem: for some norm $\| \cdot \| _\mc Y$ on $\mc Y$, Problem~\ref{eq:IP0} is generalized to a minimization problem
\begin{equation}\label{eq:IP1}
    \min_{p\in \Theta} \| y(p) - y^m \|_\mc Y.
\end{equation}
Under regularity conditions such as $\Theta$ and $\mc Y$ being Banach spaces and $y$ being a continuous map, this formulation cannot yet guarantee neither existence, nor unicity, nor stability of a solution $p*$ for every $y^m \in \mc Y$. Nonetheless, Problem~\ref{eq:IP1} is more general as it allows for solutions even if $y^{-1}(\{y^m\} )= \emptyset$, and on the practical side suggests the adoption of optimization techniques to solve IPs.

A second and often compatible approach is that of regularization techniques. 
This involves introducing additional information or constraints to stabilize the solution and make the problem well-posed. Regularization can be performed by the substitution of $y$ with some more treatable operator, or in the case of Variational Regularization by the formulation of a different minimization problem. 
Often, a regularization approach can be understood both from an operator approximation and a variational point of view: this applies to Tychonoff regularization, which can be seen as the addition of a stabilizing term to Problem~\ref{eq:IP1}
\begin{equation}\label{eq:Tycho}
    \min_{p\in \Theta} \big\| y(p) - y^m \big\|_\mc Y^2 + \lambda\big\| p - p^0 \big\|_\Theta^2,
\end{equation}
for some norm $\|\cdot\|_\Theta$ on $\Theta$, $p^0$ in $\Theta$ and $\lambda $ in $ \R^+$. \medskip

As we will see in the next section, both these techniques can be understood by adopting a statistical perspective to IPs, which considers observations corrupted by noise with random behavior and known distribution.
A parameter-to-observation relation is assumed, rendering the available observation $y^m$ a realization of a random variable $Y^m$ over the measurement space $\mc Y$.
We consider an additive noise model
\begin{equation}\label{eq:par-to-obs}
    Y^m = y(p) + N,
\end{equation}
for noise $N$ with distribution $\bb P _N$. \newline
Equation~\eqref{eq:par-to-obs} deals that for any $p$ in $\Theta$, the distribution of $Y^m$ given $p$, noted $\bb P_{Y^m \mid p}$, is given by 
\[
    \bb P_{Y^m \mid p} (A) = \bb P_N (A-y(p))
\]
for any Borel set $A$ in $\mc B (\mc Y)$. 
Note that as $p$ is not a random variable, the distribution $\bb P_{Y^m \mid p}$ is not a conditional distribution and in this context the notation is meant to highlight the dependence of the distribution on the parameter $p$. \medskip

For $\mc Y$ of finite dimension, the above equation can be expressed in terms of the probability density functions of $N$ and $Y^m$ with respect to the Lebesgue measure $m_\mc Y$.
As a function of the parameter $p$, the density of $Y^m$ given $p$ is called likelihood and given by
\begin{equation}\label{eq:likelihood}
    L(p) = \pi_{Y^m \mid p} (y^m) = \pi_N(y^m - y(p)).
\end{equation}
The likelihood of a problem is a crucial element in statistical IPs. \medskip

A natural development of the statistical approach to Inverse Problems are Bayesian Inverse Problems (BIPs). 
In the Bayesian setting, the unknown solution $p*$ of the IP is treated as a random variable $P$ and the solution of the IP then becomes the conditional distribution $\bb P_{P \mid Y^m=y^m} $ of $P$ given $Y^m=y^m$.

In full generality, the following result guarantees the possibility of formulating BIPs for arbitrary dimensionality:
\begin{thm} [Regular conditional probability]
    Let $ (\Omega, \mc F , \bb P) $ be a probability space, $\Theta$ be a separable Banach space equipped with the Borel $\sigma$-algebra $\mc B (\Theta)$, $(\mc Y, \tilde{\mc F})$ be a measurable space.
    Further, let $Y^m:\Omega \rightarrow \mc Y$ and $P : \Omega \rightarrow \Theta$ be random variables with $P \in L^1(\Omega, \bb P; \Theta) $. \newline
    Then there exists a $\bb P_{Y^m}$-a.s. unique map $\bb P_{P \mid Y^m} : \mc B (\Theta) \times \mc Y \rightarrow [0,1] $ such that :
    \begin{itemize}
        \item $\bb P_{P \mid Y^m}(\cdot, y)$ is a probability density on $\Theta$ for all $y$ in $\mc Y$;
        \item $\bb P_{P \mid Y^m}(A, \cdot)$ is measurable for all $A$ in $\mc B (\Theta)$;
        \item for all $B$ in $\sigma(Y^m)$, $A$ in $\mc B (\Theta)$, it holds that
                \[ 
                \int_B \bb P_{P \mid Y^m}(A, Y^m(\omega)) \ d\bb P(\omega)= \int_B \ind_A(P(\omega)) \ d\bb P(\omega).
                \] 
    \end{itemize}
    Such map is known as the \textbf{regular conditional probability} of $P$ given $Y^m$.
\end{thm}

This results guarantees the well-definiteness of conditional probabilities in a general setting, thus allowing for the formulation of BIPs in arbitrary dimensionality, but does not provide a direct way to formulate the posterior distribution given a parameter-to-observation relation. 
This is provided by Bayes' rule, which in generality is given by the following result from~\cite[Theorem 14]{DashtiStuart2017}:

\begin{thm}[Bayes' rule]
    Let $ (\Omega, \mc F , \bb P) $ be a probability space and $\Theta, \mc Y$ be separable Banach spaces equipped with the respective Borel $\sigma$-algebras. 
    Moreover, let $P : \Omega \rightarrow \Theta$ and $N:\Omega \rightarrow \mc Y$ be independent random variables, and $ Y^m = y(P) + N$ with $y: \Theta \rightarrow \mc Y$ a measurable map. \newline
    Assume $\bb P_{Y^m\mid P}(\cdot, p) \ll P_N$ for every $p \in \Theta$, that $\frac{d \bb P_{Y^m\mid P}(\cdot, p)}{d\bb P_N}(y) $ is $\bb P_{(P,N)}$-measurable, and that 
    \[
        \int_\Theta \frac{d \bb P_{Y^m\mid P}(\cdot, p)}{d\bb P_N}(y) \ d\bb P_P(p) > 0 \ \text{ for }  \ y, \ \bb P_N \text{-a.s.}
    \]
    holds.
    Then, the regular conditional probability $\bb P_{P\mid Y^m}(\cdot,y)$ for $P$ given $Y^m$ exists and is such that $\bb P_{P\mid Y^m}(\cdot, y) \ll \bb P _P$ $\bb P_{Y^m}$-a.s., with Radon-Nikodym derivative
    \begin{equation}\label{eq:infdimBayes}
        \frac{d\bb P_{P\mid Y^m}(\cdot, y)}{d\bb P_P}(p) = \frac{1}{Z(y)}\frac{d\bb P_{Y^m\mid P}(\cdot, p)}{d\bb P_{N}}(y).
    \end{equation}
\end{thm}

In the above theorem, $\bb P_P$ is the prior distribution of $P$ and $\bb P_{P \mid Y^m}(\cdot, y^m)$ is then the Bayesian posterior distribution of $P$ given $Y^m=y^m$.
As in~\cite[Theorem 6.31]{Stuart2010}, under certain hypothesis over the measurement space $\mc Y$, the distribution of the noise $\bb P_N$ and the forward model $y$, one has that the conditions over $\frac{d \bb P_{Y^m\mid P}(\cdot, p)}{d\bb P_N}(y)$ hold, rendering the application of the Theorem possible. \medskip

As for the scope of this work it is not necessary to work in full generality, from now on it will be assumed that the involved spaces $\Theta, \  \mc Y$ are finite dimensional Banach spaces, and that $P$ and $Y$ are random vectors that admit a joint probability density function $\pi_{P,Y}$ with respect to the Lebesgue product measure $m_\Theta \otimes m_\mc Y$. 
This allows for a more intuitive and direct formulation of the Bayesian posterior distribution of $P$ given $Y=y^m$ by exploiting the conditional probability density.

\begin{thm}[Bayes' rule with Lebesgue measure]
    Let $Y = y(P) + N$ hold, with $P$ and $N$ independent and $y: \Theta \rightarrow \mc Y$ a measurable map.
    Then the conditional density of $Y^m$ given $P$ coincides with the likelihood~\eqref{eq:likelihood} is \[
        \pi_{Y^m\mid P = p}(y^m) = \pi_{N}(y^m - y( p) )
    \] and the posterior distribution of $P$ given $Y^m=y^m$ is given by the probability density \begin{equation}\label{eq:Bayes}
        \pi_{P\mid Y^m = y^m}(p) = \frac{\pi_{N}(y^m - y( p) ) \pi_P(p)}{\int_\Theta  \pi_{N}(y^m - y( p) ) \pi_P(p) \ dm_\Theta(p)},
    \end{equation}
\end{thm}

\subsection{Solution of Inverse Problems}\label{sec:IP-sol}
ML/MAP\medskip

MCMC (mention effective size sampling)\medskip

Among other MCMC techniques, we mention Ensemble Sampling, whose implementation we will use in the numerical experiments of Section~\ref{sec:exp}.
Ensemble Sampling is a MCMC technique introduced by~\cite{GoodmanWeare} which guarantees invariancy with respect to affine transformation of the parameter space by utilizing multiple correlated chains. 
The effectiveness of sampling crucially depends on the proposal function, which in the case of Ensemble samplers are also known as moves.
As the posterior distribution could potentially exhibit multimodal behavior, we adopt the Differential-Independence Mixture Ensemble (DIME) move introduced by~\cite{Boehl}, which adapts to Ensemble sampling the Differential Evolution MCMC framework introduced by~\cite{TerBraak} and behaves efficiently on multimodal distributions. 

\subsection{Partial differential equation models} \label{sec:PDE}

\subsection{Adaptive Finite Element method} \label{sec:AdaFE}